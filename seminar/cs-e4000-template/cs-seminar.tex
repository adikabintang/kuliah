\documentclass[article]{aaltoseries}
\usepackage[utf8]{inputenc}
\usepackage{url}


\begin{document}
 
%=========================================================

\title{Digital Twin for Industrial Edge 4.0: Concepts and Tools}

\author{Adika Bintang Sulaeman% Your first and last name: do _not_ add your student number
\\\textnormal{\texttt{adika.sulaeman@aalto.fi}}} % Your Aalto e-mail address

\affiliation{\textbf{Tutor}: Prof. Hong-Linh Truong} % First and last name of your tutor

\maketitle

%==========================================================

\begin{abstract}
  Industry 4.0 is expected to be the next generation of industrial phase. Digital twin has an important role in the Industry 4.0. This paper discusses the digital twin in Industry 4.0 context as well as the supporting technologies for digital twin such as Internet of Things (IoT), cloud computing, service models, and containers.
  
\vspace{3mm}
\noindent KEYWORDS: Digital Twin, Industrial Internet of Things, Industry 4.0

\end{abstract}


%============================================================


\section{Introduction}

Industry 4.0 is the next generation of the industrial phase. With industry 4.0, it is possible to gather real-time data from the machines that run in industry and process the data into something meaningful and useful. Industry 4.0 mainly consists of three supporting technologies: IoT, Cyber-Physical Systems (CPS), and Smart Factories \cite{hermann2016design}. The combination of these technologies build interconnected devices which forms the fundamental of digital twin.

Digital twin models a physical object to a digital representation using real-time data \cite{Cheatshe3:online}. The data is gathered throughout its life-cycle and used as the source to monitor, learning, and enhance decision making. Digital twin enables engineers monitor and understand how the machines behave once it runs on production. Furthermore, engineers can analyze the data and predict the future performance of the machines.

There are some use cases of digital twin for Industry 4.0. Consider a Printed Circuit Board (PCB) printer for electronic manufacturers. The PCB printer must be very precise in their laser cutting movement, since merely one millimeter miss may lead to PCB flaws. Digital twin enables engineers and technicians to monitor and analyze the data to predict the time when the spare parts tear. Another example would be monitoring jet engine of airplanes. By analyzing data gathered in a real-time manner, engineers and technician may predict the failure in the jet systems which will lead to the reduction of airplane incidents. Furthermore, digital twin may give the feedback to the engineers who design the machine to help them realize an agile-like development system.

\subsection{Scope and Goals}
\label{sec:emphasis}
This paper aims to review the concept of digital twin for Industry 4.0 as well as the tools and challenges to implement digital twin. This paper will review the roles of IoT, software technologies such as containers and API, as well as cloud computing in digital twin.

\subsection{Structure}
\label{sec:em}
The rest of this paper is organized as follows. Section 2 presents the concepts of implementing digital twin for Industry 4.0. Section 3 discusses the tools for digital twin, i.e. the roles of IoT, containers, API, and cloud computing for digital twin. Section 4 concludes this review paper.
 

%============================================================



%============================================================


\section{Concepts}
The software architecture aspects of digital twin consists of structure and content, API and usage, Integration, and runtime environment \cite{malakuti2018architectural}.


%============================================================


\section{Conclusion}

To be added.


%============================================================


\bibliographystyle{plain}
\bibliography{cs-seminar}

\end{document}
