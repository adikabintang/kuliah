\documentclass[article]{aaltoseries}
\usepackage[utf8]{inputenc}
\usepackage{url}


\begin{document}
 
%=========================================================

\title{Digital Twins for Industrial Edge 4.0: Concepts and Tools}

\author{Adika Bintang Sulaeman% Your first and last name: do _not_ add your student number
\\\textnormal{\texttt{adika.sulaeman@aalto.fi}}} % Your Aalto e-mail address

\affiliation{\textbf{Tutor}: Prof. Hong-Linh Truong} % First and last name of your tutor

\maketitle

%==========================================================

\begin{abstract}
  Industry 4.0 is expected to be the next big phase in industry. Digital twins, defined as digital representations of physical objects such as machines, have an important role in Industry 4.0. This paper discusses the digital twin in the context of Industry 4.0 as well as the supporting technologies for digital twins such as the Internet of Things (IoT), cloud computing, service models, and containers.
  
\vspace{3mm}
\noindent KEYWORDS: Digital Twin, Industrial Internet of Things, Industry 4.0

\end{abstract}


%============================================================


\section{Introduction}

Industry 4.0 is the next big phase in industry. With industry 4.0, it is possible to gather real-time data from the machines that run in industry and process the data into something meaningful and useful. Industry 4.0 mainly consists of three supporting technologies: IoT, Cyber-Physical Systems (CPS), and Smart Factories \cite{hermann2016design}. The combination of these technologies builds interconnected devices forming digital twins.

A digital twin models a physical object by creating a digital representation using real-time data \cite{Cheatshe3:online}. The data is gathered throughout its life-cycle and used as the source to monitor, learn from, and enhance decision making. A digital twin enables engineers to monitor and understand how the machines behave once it is released and run by users. Furthermore, engineers can analyze the data and predict the future performance of the machines.

There are some use cases of digital twins for Industry 4.0. Consider a Printed Circuit Board (PCB) printer for electronic manufacturers. The PCB printer must be very precise, because the smallest error by the laser cutter may lead to PCB flaws. A digital twin enables engineers and technicians to monitor and analyze the data to predict the time when the spare parts wear out. Another example would be monitoring the jet engine of airplanes. By analyzing data gathered in real-time, engineers and technicians may predict failures in jet systems, which will lead to the reduction of airplane incidents. Furthermore, digital twins may give feedback to the engineers who design the machine to help them realize an agile development system.

The main value that the digital twin delivers is an understanding of product performance \cite{Cheatshe3:online}. By understanding performance, manufacturers may detect and understand faults better, create an effective maintenance schedule, troubleshoot machines remotely, and decide appropriate add-on services.

The digital twin has some challenges in its development and implementation. In \cite{bienhaus2017patterns}, the author has stated a number of challenges such as data consistency between the real physical assets and the digital representation, as well as connectivity and security concerns of cloud computing for digital twins. Software architectural aspects such as internal structure, APIs, integration, and runtime environment are also critical challenges for digital twins \cite{malakuti2018architectural}.

\subsection{Scope and Goals}
\label{sec:emphasis}
This paper aims to review the concept of digital twins for Industry 4.0 as well as the tools and challenges to implement digital twins. The role of the IoT, software deployment technologies, API, and cloud computing in digital twins will be the main focus of this paper.

\subsection{Structure}
\label{sec:em}
The rest of this paper is organized as follows. Section 2 presents concept of the digital twin for Industry 4.0. Section 3 discusses the tools for digital twins, i.e. the roles of IoT, containers, API, and cloud computing for the digital twin. Section 4 concludes this review paper.
 

%============================================================



%============================================================


\section{Concepts of Digital Twin}
To be added.


%============================================================


\section{Conclusion}

To be added.


%============================================================


\bibliographystyle{plain}
\bibliography{cs-seminar}

\end{document}
